\documentclass{beamer}
\usetheme[color=screen]{UniBern}

%\setbeameroption{show notes}
%\includeonlyframes{current}

\usepackage{lmodern}
\usepackage[english]{babel}
\usepackage{microtype}
\usepackage{textcomp}
\usepackage[backend=biber, style=numeric, url=false, isbn=false, maxbibnames=1, sorting=none]{biblatex}
	\addbibresource{../../Documents/library.bib}
\usepackage{graphicx}
\usepackage{tikz}
	\usetikzlibrary{arrows.meta,shapes}
\usepackage[detect-all=true, range-phrase=--, range-units=single]{siunitx}
\usepackage{csquotes}
\usepackage[absolute,overlay]{textpos} %for the \source{} command
\usepackage{gitinfo2}
\usepackage{xspace}
\usepackage{hyperref}

% Some often used abbreviations
\newcommand{\imsize}{\linewidth} % globally set image width
\newlength\imagewidth % needed for scalebars
\newlength\imagescale % needed for scalebars
\newcommand{\uct}{\si{\micro}CT\xspace} % make our life easier
\newcommand{\uaf}{\si{\micro}AngioFil\xspace} % make our life easier

% Easily fill a frame with the whole image.
% Based on http://tex.stackexchange.com/a/334758/828, http://tex.stackexchange.com/a/244103/828 and ubTitleHeight and ubFooterHeight found in the Unibe Beamer template.
\newcommand{\fullframeimage}[1]{%
	\begin{tikzpicture}[remember picture,overlay]%
		\node[xshift=0,yshift=-0.085\paperheight-0.016\paperheight/2] at (current page.center){\includegraphics[width=\paperwidth]{#1}};%
	\end{tikzpicture}%
}

% Acknowledge things in the lower right of the slide
% Based on http://tex.stackexchange.com/a/48485/828
\newcommand{\source}[1]{
	\begin{textblock*}{4cm}(8.7cm,8.6cm)%
		\begin{beamercolorbox}[ht=0.5cm,right]{framesource}%
			\tiny\usebeamerfont{framesource}\usebeamercolor[fg]{framesource} Source: {#1}%
		\end{beamercolorbox}%
	\end{textblock*}%
}

% define 'unibe' color
\definecolor{unibe}{RGB}{156,189,222}

% Show current section at begin of sections
\AtBeginSection[]{
	\begin{frame}{Outline}
	\small
	\tableofcontents[currentsection, hideothersubsections]
	\end{frame} 
}

% Define us a custom footer
\defbeamertemplate{footline}{unibe}{%
	\usebeamercolor[fg]{page number in head/foot}%
	\usebeamerfont{page number in head/foot}%
	\hspace*{0.5cm}%
	\insertshortauthor\xspace|\xspace\insertshorttitle\xspace|\xspace Version \gitAbbrevHash%
	\hspace*{\fill}
	\insertframenumber\,/\,\inserttotalframenumber%
	\hspace*{0.5cm}%
	\vskip2pt%
}
\setbeamertemplate{footline}[unibe]

% Format bibliography for beamer
% http://tex.stackexchange.com/a/10686/828
\renewbibmacro{in:}{}
% % http://tex.stackexchange.com/a/13076/828
% \AtEveryBibitem{\clearfield{title}}
\AtEveryBibitem{\clearfield{journaltitle}}
\AtEveryBibitem{\clearfield{pages}}
\AtEveryBibitem{\clearfield{volume}}
\AtEveryBibitem{\clearfield{number}}
\AtEveryBibitem{\clearfield{editors}}

% Subtitle and other informations
\title[Quantitative assessment of brain tumor vasculature]{Quantitative assessment of brain tumor radiation treatment reveals decrease in tumor-supporting vessels}
\author{David Haberthür}
\institute{Institute of Anatomy, University of Bern}
\date{July 13, 2017\\12\textsuperscript{th} microCT User Meeting}

\begin{document}
% We want no footline on the title page, http://tex.stackexchange.com/a/18829/828 helps
{%
	\setbeamertemplate{footline}{}%
\begin{frame}%
	\titlepage%
\end{frame}%
}
\setcounter{framenumber}{0}

\begin{frame}{Contents}
	\tableofcontents
\end{frame}

\renewcommand{\imsize}{0.95\textheight}
\begin{frame}{Computed tomography}
	\begin{columns}
		\begin{column}{0.7\linewidth}
			\includegraphics[width=\imsize]{img/CT_PRINCI_PB}
				\source{\href{https://commons.wikimedia.org/wiki/File:CT_PRINCI_PB.jpg}{enwp.org/tomography}}	
		\end{column}
		\begin{column}{0.4\linewidth}
			\begin{enumerate}
				\item Object
				\item Parallel light source
				\item Screen
				\item Transmitted beam
				\item Datum circle
				\item Origin
				\item 1D image
			\end{enumerate}
		\end{column}
	\end{columns}
\end{frame}

\begin{frame}
	\frametitle{Some image}
	\fullframeimage{img/Montage}
\end{frame}

\begin{frame}
	\frametitle{Summary}
	\begin{itemize}
		\item \uct is a powerful method to get both pretty and meaningful insight into a broad kind of samples
		\item Non-destructiveness makes it a complementary method to established assessment methods (Stereology, SEM, Histology)
		\item Three-dimensional, numerical data enables quantitative assessment of sample properties
		\item[]
		\pause
		\item Come and ask us if you want to look into your samples!
	\end{itemize}
\end{frame}

\begin{frame}
	\frametitle{Thanks}
	Here is just some text without too much information.
 \end{frame}

\begin{frame}
	\frametitle{Thanks}
	\begin{itemize}
		\item Topographic and clinical Anatomy, namely
		\begin{itemize}
			\item Ruslan Hlushchuk
			\item Marine Potez
			\item Audrey Bouchet
			\item Valentin Djonov
		\end{itemize}
		\item ESRF Grenoble
		\item SNF
		\pause
		\item You, for listening
		\item[]
		\pause
		\item Questions?
	\end{itemize}
\end{frame}

\begin{frame}[allowframebreaks]
	\frametitle{References}
	\renewcommand*{\bibfont}{\scriptsize}
	\setbeamertemplate{bibliography item}{\insertbiblabel}
	\printbibliography
\end{frame}

\end{document}